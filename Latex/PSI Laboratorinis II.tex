\documentclass[oneside]{VUMIFPSkursinis}
\usepackage{algorithmicx}
\usepackage{algorithm}
\usepackage{algpseudocode}
\usepackage{amsfonts}
\usepackage{float}
\usepackage{amsmath}
\usepackage{bm}
\usepackage{caption}
\usepackage{color}
\usepackage{float}
\usepackage{graphicx}
\usepackage{listings}
\usepackage{subfig}
\usepackage{tabularx}
\usepackage{wrapfig}
\newcolumntype{P}[1]{>{\centering\arraybackslash}p{#1}}
\usepackage[%  
    colorlinks=true,
    linkcolor=black
]{hyperref}
\university{Vilniaus universitetas}
\faculty{Matematikos ir informatikos fakultetas}
\department{Programų sistemų katedra}
\papertype{Laboratorinis darbas II}
\title{Automatinė ūkio valdymo sistema}
\titleineng{Automatic farm management system}
\status{2 kurso 3 grupės studentai}
\author{Matas Savickis}
\secondauthor{Justas Tvarijonas}  
\thirdauthor{Greta Pyrantaitė}   
\fourthauthor{Rytautas Kvašinskas}
\supervisor{Karolis Petrauskas, Doc., Dr.}
\date{Vilnius – \the\year}


\bibliography{bibliografija}

\begin{document}
\maketitle


\centering

 
\tableofcontents


\section{Įvadas}
\subsection{Tikslas}
Šiuo dokumentu siekiame detaliai perteikti Automatinės ūkio valdymo sistemos aprašą. Dokumente pateikti sistemos tikslai, jų įgyvendinimas, sąsajos su išore. Taip pat pateikiami funkciniai ir nefunkciniai sistemos reikalavimai. Šis dokumentas turėtų padėti susipažinti su sistema programuotojams, testuotojams, investuotojams bei vartotojams norintiems labiau įsigilinti į programos veikimą.
\subsection{Dokumento konvensija}
\begin{itemize}
	\item Dokumentas struktūrizuotas pagal IEEE 830 Software requirements šabloną.
	\item Dokumentas formatuotas prisilaikant kursinio darbo metodinius reikalavimus.
\end{itemize}
\subsection{Dokumento skaitytojai}
\begin{itemize}
	\item Užsakovas - dokumento informacija leis išsiaiškinti kokius funkcionalus programa atliks ir kokių ne. Šis dokumentas padės išvengti neaiškumų bendraujant su sistemos kūrėjais.
	\item Projekto vadovas - dokumentas leis išvengti nesutarimų su užsakovų. Taip pat šio dokumento pagalba bus galima pasakyti koks apytiksliai biudžetas bus reikalingas įgyvendintas visus funkcionalumus, kiek laiko tai gali užtrukti ir kokių kitų resursų gali prireikti siekiant tinkamai įvykdyti projektą.
	\item Projektuotojas - dokumento informacija padės išsiaiškinti kokius technologinius ir architektūrinius sprendimus reiks priimti siekiant užtikrinti sistemos įgyvendinimą.
	\item Testuotojas - dokumentas leis suprasti koks yra numatytas programos veikimas ir kas yra nenumatytos klaidos, bei nenumatytas programos veikimas.
	\item Teisininkas - iškilus teisiniams nesklandumams tarp užsakovo ir darbų vykdytojų dokumentas leis įvertinti ar buvo įvykdyti visi funkcionalumai užsibrėžti darbų vykdytojų. Iškilus kitiems teisiniams nesklandumams, tokiems kaip ar programa nepažeidžia įstatymų dokumentas leis išsiaiškinti, kurios sistemos dalys buvo sukurtos planuotai o kurios ne.
	\item Naudotoja - dokumentas suteiks detalenią informaciją apie sistemą vartotojams norintiems pagilinti žianis apie tai kaip veikia programa.
	\item Programuotojas - dokumentas leis naujiems programuotojams susipažinti su bendru sistemos veikimu ir lengviau bei greičiau prisidėti prie sistemos tobulinimo. ir palaikymo.
	\item Rinkodaros personalas - dokumentas leis išskirti sistemos funkcionalumus ir lengiau juos pateikti vartotojams reklamuose bei kituose rinkodarinėse kampanijose.
\end{itemize}
\pagebreak
\subsection{Produkto apimtis} Automatinė ūkio valdymo sistema yra produktas skirta modernizuoti ūkio valdymą. Sistema leidžia vartotojui nuotoliniu būdu stebėti gyvulių parametrus, sekti turimus žmogiškuosius ir tertinius išteklius. Sistema taip pat leidžia valdyti išteklius, samdyti darbuotojus, pirkti ir parduoti techniką, stebint rinkos kainas parduoti turimą derlių. Pagrindinis sistemos privalumas tas, kad ūkininkui nebūtina būti savo valdomoje teritorijoje norit užtikrinti ūkio valdymą. Su šia sistema ūki galima valdyti su išnamuoju telefonu ar kompiuteriu iš betokios vietos kur yra interneto ryšys. Sistema pritaikyta tiek mažiems tiek dideliems ūkiams valdyti. 
\subsection{Nuorodos}
\begin{itemize}
	\item Diagramoms braižyti naudojome \url{www.draw.io} bei \url{www.planttext.com}
	\item Dokumentas parašytas paga IEEE 830 šabloną \url{https://en.wikipedia.org/wiki/Software_requirements_specification}
	\item Panaši programa jau egzistuojanti rinkoje \url{www.farmis.lt}
	\item Kursinio darbo metodiniai nurodymai \url{http://www.mif.vu.lt/katedros/se/Studentams/KURSINIO%20DARBO%20METODINIAI%20NURODYMAI%202011_AL.pdf}
\end{itemize}

\section{Bendras produkto aprasymas}
\subsection{Produkto perspektyva}
\subsection{Produkto funkcionalumas}
\subsection{Vartotojų klasės ir charakteristikos}
\subsection{Vykdymo aplinka}
\subsection{Dizaino ir implementacijos apribojimai}
\subsection{Prielaidos ir priklausomybės}

\section{Išoriniai sąsajos reikalavimai}
\subsection{Vartotojo sąsajos reikalavimai}
\subsection{Techninės įrangos sąsajos reikalavimai}
\subsection{Programinės įrangos sąsąjos reikalavimai}
\subsection{Komunikavimo sąsajos reikalavimai}

\section{Produkto funkcijos}
\subsection{Pirma funkcija}

\section{Kiti nefunkciniai reikalavimai}
\subsection{Našumo reikalavimai}
\subsection{Saugos reikalavimai}
\subsection{Saugumo reikalavimai}
\subsection{Sistemos kokybės atributai}
\subsection{Biznio taisyklės}

\section{Kiti reikalavimai}
\sectionnonum{Žodynas}
\sectionnonum{Analizės modeliai}
\end{document}