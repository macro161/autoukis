\documentclass[oneside]{VUMIFPSkursinis}
\usepackage{algorithmicx}
\usepackage{algorithm}
\usepackage{algpseudocode}
\usepackage{amsfonts}
\usepackage{float}
\usepackage{amsmath}
\usepackage{bm}
\usepackage{caption}
\usepackage{color}
\usepackage{float}
\usepackage{graphicx}
\usepackage{listings}
\usepackage{subfig}
\usepackage{tabularx}
\usepackage{wrapfig}
\newcolumntype{P}[1]{>{\centering\arraybackslash}p{#1}}
\usepackage[%  
    colorlinks=true,
    linkcolor=black
]{hyperref}
\university{Vilniaus universitetas}
\faculty{Matematikos ir informatikos fakultetas}
\department{Programų sistemų katedra}
\papertype{Laboratorinis darbas II}
\title{Automatinė ūkio valdymo sistema}
\titleineng{Automatic farm management system}
\status{2 kurso 3 grupės studentai}
\author{Matas Savickis}
\secondauthor{Justas Tvarijonas}  
\thirdauthor{Greta Pyrantaitė}   
\fourthauthor{Rytautas Kvašinskas}
\supervisor{Karolis Petrauskas, Doc., Dr.}
\date{Vilnius – \the\year}


\bibliography{bibliografija}

\begin{document}
\maketitle


\centering

 
\tableofcontents


\section{Įvadas}
\subsection{Tikslas}
Šiuo dokumentu siekiame detaliai perteikti Automatinės ūkio valdymo sistemos aprašą. Dokumente pateikti sistemos tikslai, jų įgyvendinimas, sąsajos su išore. Taip pat pateikiami funkciniai ir nefunkciniai sistemos reikalavimai. Šis dokumentas turėtų padėti susipažinti su sistema programuotojams, testuotojams, investuotojams bei vartotojams norintiems labiau įsigilinti į programos veikimą.
\subsection{Dokumento konvensija}
\begin{itemize}
	\item Dokumentas struktūrizuotas pagal IEEE 830 Software requirements šabloną.
	\item Dokumentas formatuotas prisilaikant kursinio darbo metodinius reikalavimus.
\end{itemize}
\subsection{Dokumento skaitytojai}
\begin{itemize}
	\item Užsakovas
	\item Projekto vadovas
	\item Projektuotojas
	\item Testuotojas
	\item Teisininkas
	\item Naudotoja
\end{itemize}
\subsection{Produkto apimtis}
\subsection{Nuorodos}

\section{Bendras produkto aprasymas}
\subsection{Produkto perspektyva}
\subsection{Produkto funkcionalumas}
\subsection{Vartotojų klasės ir charakteristikos}
\subsection{Vykdymo aplinka}
\subsection{Dizaino ir implementacijos apribojimai}
\subsection{Prielaidos ir priklausomybės}

\section{Išoriniai sąsajos reikalavimai}
\subsection{Vartotojo sąsajos reikalavimai}
\subsection{Techninės įrangos sąsajos reikalavimai}
\subsection{Programinės įrangos sąsąjos reikalavimai}
\subsection{Komunikavimo sąsajos reikalavimai}

\section{Produkto funkcijos}
\subsection{Pirma funkcija}

\section{Kiti nefunkciniai reikalavimai}
\subsection{Našumo reikalavimai}
\subsection{Saugos reikalavimai}
\subsection{Saugumo reikalavimai}
\subsection{Sistemos kokybės atributai}
\subsection{Biznio taisyklės}

\section{Kiti reikalavimai}
\sectionnonum{Žodynas}
\sectionnonum{Analizės modeliai}
\end{document}