\documentclass[oneside]{VUMIFPSkursinis}
\usepackage{algorithmicx}
\usepackage{algorithm}
\usepackage{algpseudocode}
\usepackage{amsfonts}
\usepackage{float}
\usepackage{amsmath}
\usepackage{bm}
\usepackage{caption}
\usepackage{color}
\usepackage{float}
\usepackage{graphicx}
\usepackage{listings}
\usepackage{subfig}
\usepackage{tabularx}
\usepackage{wrapfig}
\newcolumntype{P}[1]{>{\centering\arraybackslash}p{#1}}
\usepackage[%  
    colorlinks=true,
    linkcolor=black
]{hyperref}
\university{Vilniaus universitetas}
\faculty{Matematikos ir informatikos fakultetas}
\department{Programų sistemų katedra}
\papertype{Laboratorinis darbas II}
\title{Automatinė ūkio valdymo sistema}
\titleineng{Automatic farm management system}
\status{2 kurso 3 grupės studentai}
\author{Matas Savickis}
\secondauthor{Justas Tvarijonas}  
\thirdauthor{Greta Pyrantaitė}   
\fourthauthor{Rytautas Kvašinskas}
\supervisor{Karolis Petrauskas, Doc., Dr.}
\date{Vilnius – \the\year}


\bibliography{bibliografija}

\begin{document}
\maketitle

\tableofcontents
\centering

\section{Įvadas}
\subsection{Tikslas}
\subsection{Dokumento skaitytojai}
\begin{itemize}
	\item Užsakovas
	\item Projekto vadovas
	\item Projektuotojas
	\item Testuotojas
	\item Teisininkas
\end{itemize}

\section{Funkciniai reikalavimai}
\subsection{Duomenų suvedimas}

\begin{table}[htbp]
	\begin{tabularx}{1\textwidth}{ |X|P{3cm }| }
       	          \hline
           	Reikalavimas &  Prioritetas (1-10)  \\   \hline 
        		Ūkininkui turi būti leista pasirinkti mąstelį &  10  \\   \hline
         		Ūkininkui turi būti leista žymėti teritorijas &  10  \\   \hline
        		Darbuotojui turi būti leista suvesti atliktus darbus & 10  \\   \hline
        		Ūkininkui turi būti leista keisti darbuotojo duomeis & 8 \\ \hline
        		Visi naudotojai turi turėti galimybę suvesti gyvūmų duomenis & 10 \\ \hline
	\end{tabularx}
\end{table}

\subsection{Duomenų peržiūra}

\begin{table}[htbp]
	\begin{tabularx}{1\textwidth}{ |X|P{3cm }| }
       	           \hline
       	            Reikalavimas &  Prioritetas (1-10)  \\   \hline 
        		 Visi naudotojai turi turėti galimybę peržiūrėti orų prognozes pasirinktai teritorijai & 8 \\ \hline
        		 Visi naudotojai turi turėti galimybę peržiūrėti Guvūnų duomenis & 10 \\ \hline
        		 Ūkininkui turi būti leista peržiūrėti  esamus resursus & 10 \\ \hline
        		 Adminas ir Ūkininkas turi turėti galimybę pamatyti detektorių būsenas & 9 \\ \hline
        		 Adminas ir Ūkininkas turi turėti galimybę sekti transporto judėjimą & 9 \\ \hline
        		 Adminas ir Ūkininkas turi turėti galimybę gauti įvykių žurnalą & 10 \\ \hline
	\end{tabularx}
\end{table}

\subsection{Vartotojo registracija}

\begin{table}[htbp]
	\begin{tabularx}{1\textwidth}{ |X|P{3cm }| }
       	          \hline
           	Reikalavimas &  Prioritetas (1-10)  \\   \hline 
        		Darbuotojui turi būtų leista pateikti užklausą registruotis  &  10  \\   \hline
         		Darbuotojui turi būtų leista Užpildyti registracijos formą  &  10  \\   \hline
        		Darbuotojui turi būtų leista patvirtinti registraciją  & 10  \\   \hline
        		Ūkininkas turi galėti vartotojui suteikti teisių & 10 \\ \hline
        		Naujas vartotojas turi būti aktyvuojamas & 10 \\ \hline
	\end{tabularx}
\end{table}

\subsection{Finansų tvarkymas}

\begin{table}[htbp]
	\begin{tabularx}{1\textwidth}{ |X|P{3cm }| }
       	          \hline
           	Reikalavimas &  Prioritetas (1-10)  \\   \hline 
        		Ūkininkas turi galėti pamatyti pasirinkto laikotarpio planuojamą pelną  &  9  \\   \hline
         		Ūkininkas turi galėti pamatyti kapitalo apskaitą &  9  \\   \hline
        		Ūkininkas turi galėti pamatyti pasirinkto laikotarpio išlaidas  & 9  \\   \hline
        		Ūkininkui turi būti leista išmokėti algas darbuotojams & 10 \\ \hline
	\end{tabularx}
\end{table}

\section{Nefunkciniai reikalavimai}
\subsection{Naudojimo saugos reikalavimai}
\subsection{Duomenų saugos reikalavimai}
\subsection{Kokybės reikalavimai}

\subsection{Veikimo principai}
\section{Išoriniai sąsajos reikalavimai}
\subsection{Vartotojo sąsajos reikalavimai}

\end{document}