\documentclass[oneside]{VUMIFPSkursinis}
\usepackage{algorithmicx}
\usepackage{algorithm}
\usepackage{algpseudocode}
\usepackage{amsfonts}
\usepackage{float}
\usepackage{amsmath}
\usepackage{bm}
\usepackage{caption}
\usepackage{color}
\usepackage{float}
\usepackage{graphicx}
\usepackage{listings}
\usepackage{subfig}
\usepackage{tabularx}
\usepackage{wrapfig}
\newcolumntype{P}[1]{>{\centering\arraybackslash}p{#1}}
\usepackage[%  
    colorlinks=true,
    linkcolor=black
]{hyperref}
\university{Vilniaus universitetas}
\faculty{Matematikos ir informatikos fakultetas}
\department{Programų sistemų katedra}
\papertype{Laboratorinis darbas II}
\title{Automatinė ūkio valdymo sistema}
\titleineng{Automatic farm management system}
\status{2 kurso 3 grupės studentai}
\author{Matas Savickis}
\secondauthor{Justas Tvarijonas}  
\thirdauthor{Greta Pyrantaitė}   
\fourthauthor{Rytautas Kvašinskas}
\supervisor{Karolis Petrauskas, Doc., Dr.}
\date{Vilnius – \the\year}


\bibliography{bibliografija}

\begin{document}
\maketitle


\centering

 
\tableofcontents


\section{Įvadas}
\subsection{Tikslas}
Šiuo dokumentu siekiame detaliai perteikti Automatinės ūkio valdymo sistemos aprašą. Dokumente pateikti sistemos tikslai, jų įgyvendinimas, sąsajos su išore. Taip pat pateikiami funkciniai ir nefunkciniai sistemos reikalavimai. Šis dokumentas turėtų padėti susipažinti su sistema programuotojams, testuotojams, investuotojams bei vartotojams norintiems labiau įsigilinti į programos veikimą.
\subsection{Dokumento konvensija}
\begin{itemize}
	\item Dokumentas struktūrizuotas pagal IEEE 830 Software requirements šabloną.
	\item Dokumentas formatuotas prisilaikant kursinio darbo metodinius reikalavimus.
\end{itemize}
\subsection{Dokumento skaitytojai}
\begin{itemize}
	\item Užsakovas - dokumento informacija leis išsiaiškinti kokius funkcionalus programa atliks ir kokių ne. Šis dokumentas padės išvengti neaiškumų bendraujant su sistemos kūrėjais.
	\item Projekto vadovas - dokumentas leis išvengti nesutarimų su užsakovų. Taip pat šio dokumento pagalba bus galima pasakyti koks apytiksliai biudžetas bus reikalingas įgyvendintas visus funkcionalumus, kiek laiko tai gali užtrukti ir kokių kitų resursų gali prireikti siekiant tinkamai įvykdyti projektą.
	\item Projektuotojas - dokumento informacija padės išsiaiškinti kokius technologinius ir architektūrinius sprendimus reiks priimti siekiant užtikrinti sistemos įgyvendinimą.
	\item Testuotojas - dokumentas leis suprasti koks yra numatytas programos veikimas ir kas yra nenumatytos klaidos, bei nenumatytas programos veikimas.
	\item Teisininkas - iškilus teisiniams nesklandumams tarp užsakovo ir darbų vykdytojų dokumentas leis įvertinti ar buvo įvykdyti visi funkcionalumai užsibrėžti darbų vykdytojų. Iškilus kitiems teisiniams nesklandumams, tokiems kaip ar programa nepažeidžia įstatymų dokumentas leis išsiaiškinti, kurios sistemos dalys buvo sukurtos planuotai o kurios ne.
	\item Naudotoja - dokumentas suteiks detalenią informaciją apie sistemą vartotojams norintiems pagilinti žianis apie tai kaip veikia programa.
	\item Programuotojas - dokumentas leis naujiems programuotojams susipažinti su bendru sistemos veikimu ir lengviau bei greičiau prisidėti prie sistemos tobulinimo. ir palaikymo.
	\item Rinkodaros personalas - dokumentas leis išskirti sistemos funkcionalumus ir lengiau juos pateikti vartotojams reklamuose bei kituose rinkodarinėse kampanijose.
\end{itemize}
\pagebreak
\subsection{Produkto apimtis} Automatinė ūkio valdymo sistema yra produktas skirta modernizuoti ūkio valdymą. Sistema leidžia vartotojui nuotoliniu būdu stebėti gyvulių parametrus, sekti turimus žmogiškuosius ir tertinius išteklius. Sistema taip pat leidžia valdyti išteklius, samdyti darbuotojus, pirkti ir parduoti techniką, stebint rinkos kainas parduoti turimą derlių. Pagrindinis sistemos privalumas tas, kad ūkininkui nebūtina būti savo valdomoje teritorijoje norit užtikrinti ūkio valdymą. Su šia sistema ūki galima valdyti su išnamuoju telefonu ar kompiuteriu iš betokios vietos kur yra interneto ryšys. Sistema pritaikyta tiek mažiems tiek dideliems ūkiams valdyti. 
\subsection{Nuorodos}
\begin{itemize}
	\item Diagramoms braižyti naudojome \url{www.draw.io} bei \url{www.planttext.com}
	\item Dokumentas parašytas paga IEEE 830 šabloną \url{https://en.wikipedia.org/wiki/Software_requirements_specification}
	\item Panaši programa jau egzistuojanti rinkoje \url{www.farmis.lt}
	\item Kursinio darbo metodiniai nurodymai \url{http://www.mif.vu.lt/katedros/se/Studentams/KURSINIO%20DARBO%20METODINIAI%20NURODYMAI%202011_AL.pdf}
	\item AutoMtinis ūkio technikos valdymas \url{https://www.asirobots.com/platforms/mobius/}
	\item Buhalterija ir sąskaitos \url{https://www.manager.io/}
\end{itemize}

\section{Bendras produkto aprasymas}
\subsection{Produkto perspektyva}
Sistema yra nauja idėja skirta modernizuoti ūkio valdymą. Produktas skirtas konkuruoti su rinkoje jau egzistuojančia Farmis ūkio valdymo sistema. Mūsų kuriama sistema papildys konkurentų jau turimą sistemą naujais funkcionalumais kurie turėtų dominti ūkinikus norinčius labiau moternizuoti ir automatizuoti savo turimą ūkį ir verslą. 
\subsection{Produkto funkcionalumas}
\begin{itemize}
	\item Gyvūnų sveikatos, lokacijos bei kitų paramterų sekimas
	\item Ūkio technikos resursų sekimas, pirkimas ir pardavimas
	\item Žemės parametrų sekimas
	\item Orų prognozės sekimas
	\item Žemės parametrų spėjimas
	\item Gyvūnų maisto išteklių sekimas
	\item Automatinis gyvūnų maitinimas
	\item Automatinis maisto užsakymas
	\item Ūkio technikos sekimas realiu laiku
	\item Ūkio technikos valdymas nuotolinių būdu realiu laiku
	\item Autonominis ūkio technikos veikimas
	\item Ūkininko valdomos teritorijos žymėjimas sutartiniais ženklais
	\item Ligū žemėlapis kaimyninėse teritorijose
	\item Sąskaitų išrašymas
	\item Darbuotojų samdymas
	\item Potencialaus pelno skaičiavimas
	\item Derliaus sekimas
	\item Buhalterijos tvarkymas
	\item Rinkos kainų sekimas
	\item Automatinis žemės laistymas
	\item Greitosios pagalbos iškvietimas
	\item Apsaugos tarnybos iškvietimas
	\item Agronomo iškvietimas
	\item Ataskaitos apie ūkį sudarymas
	\item Darbų sąsašo sudarymas
	\item Žolių ir ligų katalogas
	\item Ūkio chemijos sąrašas su aprašymais ir kainomis
\end{itemize}
\subsection{Vartotojų klasės ir charakteristikos}
Sistemas bus naudojama tiek mažų tiek didelių ūkių savininkų kurie nori automatizuoti savo ūkio valdymą. Žinoma visų funkcijų implremrentavimas į ūki kainuoja nepigiai todėl didiesiams ūkininkams ši sistema tūrėtų atrodyti patrauklesnė nei mažiesiams. Tačiau kaikurie funkcionalumai įgyvendinami gan lengvai ir nebrangiai. Kaikuriais sistemos funkcionalumais gali naudotis ir ūkio darbuotojai.
\subsection{Vykdymo aplinka}
Duomenys bus saugomi serveryje, duomenų bazėje. Bus naudojamos šios technologijos:
	\begin{itemize}
		\item PostgreSQL
		\item JAVA
	\end{itemize}
Andoid aplikacija bus sukurta su šiomis technologijomis:
	\begin{itemize}
		\item JAVA
		\item Android SDK
	\end{itemize}
Kompiuterio aplikacija bus sukurta su šiomis technologijomis:
	\begin{itemize}
		\item JAVA
	\end{itemize}
Žėmės parametrai ir gyvūnų lokacija bus stebima šiomis technologijomis:
	\begin{itemize}
		\item Arduino
		\item Arduino GPS modual
		\item Arduino moisture sensor
		\item Photosensor
	\end{itemize}
Automatiniam ūkio technikos valdymui naudosimes šios technologijomis:
	\begin{itemize}		
		\item ASI Mobius
	\end{itemize}
\subsection{Dizaino ir implementacijos apribojimai}
Pagrindinis apribojimas norit įgyvendinti automatinį technikos valdymą yra sutartis su ASI kompanija dėl Mobius technologijos naudojimo. Šis sistemos funkcionalumas priklauso nuo galimybės susitarti dėl technologijos naudojimo ir nuo to ar ASI ir toliau palaikys savo technologijos palaikymą. Kitas technologinis apribojimas yra skirtingos išmaniųjų telefonų versijos. Norint užtikrinti kad sistema veiktų ant daugumos Android operacinės sistemso versijų reiks atsižvelgti į tas operacines sistemas. 


\subsection{Prielaidos ir priklausomybės}
Tinkamas programos programos veikimas priklauso nuo daugybės faktorių. Sistema iš esmės apjungia keletą išorinių komponentų į vieną itegralią sistemą. Nors sutrikus ryšiui su išoriniai komponentais kaikurios sistemos funkcijos būtų nepasiekiamos tačiau visa sistema nenutrauktų darbo. Automatinis technikos valdymas priklauso nuo ASI kompanijos programos palaikymo. Duomenų gavimas priklauso nuo to ar bus interneto ryšys ir ar pavyks pasiekti serverį, kad programa gautų prieiga prie duomenų bazės. Orų prognozė priklauso nuo to ar orų tarnyba toliau teiks prieeiga prie savo duomenų. Sąskaitų ir buhalterijos funkcionalumas priklauso nuo to \url{www. ar manager.io} toliau palaikys savo programos veikimą ir leis ja naudotis nemokamai.
\subsection{Vartotojo dokumentacija}
Naujiems vartotojams supažindinti su sistema bus sukurta video pamokų Youtube svetainėje. Neturintiems galimybės peržiūrėti vaizdo įrašo supažindinimas su programa bus integruotas į pačia programą. Vartotojas galės pasirinkti ar nori gauti pagalbines žinutes. Šias žinutes vartotojas galės išjungti nustatymuose. 

\section{Išoriniai sąsajos reikalavimai}
\subsection{Vartotojo sąsajos reikalavimai}
\subsection{Techninės įrangos sąsajos reikalavimai}
\subsection{Programinės įrangos sąsąjos reikalavimai}
\subsection{Komunikavimo sąsajos reikalavimai}

\section{Produkto funkcijos}

\subsection{Gyvūnų sveikatos, lokacijos bei kitų parametrų sekimas}
\subsubsection{Pagrindinis scenarijus}
	Vartotojas prisijungia prie sistemos, pasirenka skiltį "Gyvūnų sveikatos ir lokacijos sekimas", įveda gyvūno duomenis. Tada vartotojas išsaugo įvestus duomenis į sąrašą, kurį jis gali peržiūrėti, redaguoti.
\subsubsection{Alternatyvus scenarijus (Toks ID jau egzistuoja)}
	Vartotojas prisijungia prie sistemos, pasirenka skiltį "Gyvūnų sveikatos ir lokacijos sekimas", įveda gyvūno duomenis. Tada vartotojas pasirenka išsaugoti įvestus duomenis į sąrašą, tačiau gaunama klaida, jog gyvūnas su įvestu ID jau egzistuoja sąraše. Sistema pateikia vartotojui du pasirinkimus: pakeisti ID arba pakeisti sąraše egzistuojančio elemento, su tuo pačiu ID, duomenis.
	Pasirinkus pakeisti ID, vartotojas įveda naują ID ir pasirenka išsaugoti duomenis. 
	Pasirinkus pakeisti sąraše egzistuojančio elemento, su tuo pačiu ID, duomenis, visi sąraše buvę duomenys pakeičiami naujais įvestais duomenimis ir išsaugomi sąraše.
\subsubsection{Alternatyvus scenarijus (Nepavyko prisijungti prie sistemos)}
	Vartotojas pateikia prisijungimo duomenis ir spaudžia mygtuką "Prisijungti", tačiau gaunama klaida, jog nepavyko prisijungti prie sistemos. Į ekraną išvedamas klaidos pranešimas, ir grįžtama i pagrindinį langą, kuriame vartotojas vėl gali bandyti prisijungti arba užsiregistruoti.
\subsubsection{Funkciniai reikalavimai}
\begin{table}[htbp]
	\begin{tabularx}{1\textwidth}{ |P{2.5cm}|X|P{3cm }| } \hline
           	Nr. & Reikalavimas &  Prioritetas (1-10)  \\   \hline 
         	FR-1.01 & Sistema leidžia vartotojui dirbti su gyvūnų duomenimis & 10  \\   \hline
		FR-1.01.01 & Įvesti duomenis & 8 \\ \hline
		FR-1.01.02 & Peržiūrėti duomenis & 8 \\ \hline
		FR-1.01.03 & Redaguoti duomenis & 8 \\ \hline
		FR-1.01.04 & Išsaugoti duomenis & 8 \\ \hline
        	FR-1.02 & Sutampus raktiniams duomenims, sistema leidžia juos apdoroti & 9   \\   \hline
		FR-1.02.01 & Pakeisti raktinius duomenis, kad jie skirtųsi & 8 \\ \hline
		FR-1.02.02 & Pakeisti išsaugotus duomenis naujai įvestais duomenimis & 8 \\ \hline
	\end{tabularx}
\end{table}

\subsection{Ūkio technikos resursų sekimas}
\subsubsection{Pagrindinis scenarijus}
	Vartotojas pasirenka skiltį "Technikos sekimas realiu laiku", sistema parodo sąrašą, kuriame rodoma ūkio technika, jos užimtumas, būklė, darbo tvarkaraštis. Vartotojas pasirenka stebėti techniką realiu laiku, sistema parodo žemėlapį, kuriame matoma kur yra visos technikos priemonės, kiek apytiksliai laiko dirbs, ir ką darys toliau.
\subsubsection{Alternatyvus scenarijus(Nesukonfiguruoti priemonių sekikliai)}
	Vartotojas pasirenka skiltį "Technikos sekimas realiu laiku", sistema parodo sąrašą, kuriame rodoma ūkio technika, jos užimtumas, tačiau prie būklės rašoma klaida, jog nesukonfiguruotas priemonės sekiklis. Vartotojui sistema siūlo iškviesti sistemos administratorių, kuris sukonfiguruoja sekiklius ir nukreipia nukreipia vartotoją į pagrindinį meniu.
\subsubsection{Alternatyvus scenarijus(Pasirinkta priemonė sugedus)}
	Vartotojas pasirenka skiltį "Technikos sekimas realiu laiku", sistema parodo sąrašą, kuriame rodoma ūkio technika, jos užimtumas, tačiau prie būklės rašoma klaida, jog priemonė sugedus. Sistema iškviečia mechaniką, išveda pranešimą, jog mechanikas iškviestas ir nukreipia vartotoją į pagrindinį meniu.
\subsubsection{Funkciniai reikalavimai}
\begin{table}[htbp]
	\begin{tabularx}{1\textwidth}{ |P{2.5cm}|X|P{3cm }| }  \hline
           	Nr. & Reikalavimas &  Prioritetas (1-10)  \\   \hline 
         		FR-2.01 & Sistema leidžia vartotojui stebėti techniką & 10  \\   \hline
		FR-2.01.01 & Stebėti technikos būklę & 8 \\ \hline
		FR-2.01.02 & Peržiūrėti priemonės tvarkaraštį & 8 \\ \hline
		FR-2.01.03 & Stebėti priemonės veiklą realiu laiku & 8 \\ \hline
        		FR-2.02 & Sugedus priemonės sekikliui, programa leidžia iškviesti sistemos administratorių & 9   \\   \hline
			FR-2.03 & Sugedus priemonei, sistema iškviečia mechaniką & 9 \\ \hline
	\end{tabularx}
\end{table}

\subsection{Ūkio technikos resusrsų pirkimas}
\subsubsection{Pagrindinis scenarijus}
	Vartotojas pasirenka skiltį "Technikos sekimas realiu laiku", sistema parodo sąrašą, kuriame rodoma ūkio technika, jos užimtumas, būklė, darbo tvarkaraštis. Vartotojas pasirenka pirkti naują technikos priemonę, sistema nuveda jį į puslapį, kuriame yra pasiūlymai jo norimai technikos rūšiai pirkti.
\subsubsection{Alternatyvus scenarijus(Nepavyko rasti skelbimų norimai priemonei)}
	Vartotojas pasirenka skiltį "Technikos sekimas realiu laiku", sistema parodo sąrašą, kuriame rodoma ūkio technika, jos užimtumas, būklė, darbo tvarkaraštis. Vartotojas pasirenka pirkti naują technikos priemonę, sistema nuveda jį į puslapį, kuriame yra pasiūlymai jo norimai technikos rūšiai pirkti, tačiau pasirinktai priemonei įiuo metu nėra sukurtų jokių skelbimų. Sistema išveda pranešimą, jog pasirinktos priemonės pasiūlymų nėra.
\subsubsection{Alternatyvus scenarijus(Nepavyko prisijungti prie serverio)}
	Vartotojas pasirenka skiltį "Technikos sekimas realiu laiku", sistema parodo sąrašą, kuriame rodoma ūkio technika, jos užimtumas, būklė, darbo tvarkaraštis. Vartotojas pasirenka pirkti naują technikos priemonę, tačiau sistemai nepavyko prisijungti prie serverio. Sistema išveda klaidos pranešimą ir nukreipia vartotoją į pagrindinį meniu.
\subsubsection{Funkciniai reikalavimai}
\begin{table}[htbp]
	\begin{tabularx}{1\textwidth}{ |P{2.5cm}|X|P{3cm }| }  \hline
           	Nr. & Reikalavimas &  Prioritetas (1-10)  \\   \hline 
         		FR-3.01 & Sistema leidžia vartotojui pirkti techniką & 10  \\   \hline
		FR-3.01.01 & Peržiūrėti skelbimus internete & 8 \\ \hline
        		FR-3.02 & Neradus norimos priemonės skelbimų, sistema išveda pranešimą & 8   \\   \hline
			FR-3.03 & Įvykus klaidai prisijungiant prie serverio, sistema išveda klaidos pranešimą ir nukreipia vartotoją į pagrindinį meniu & 8 \\ \hline
	\end{tabularx}
\end{table}

\subsection{Ūkio technikos resursų pardavimas}
\subsubsection{Pagrindinis scenarijus}
	Vartotojas pasirenka skiltį "Technikos sekimas realiu laiku", sistema parodo sąrašą, kuriame rodoma ūkio technika, jos užimtumas, būklė, darbo tvarkaraštis. Vartotojas pasirenka parduoti technikos priemonę, sistema patikrina, ar ši priemonė yra laisva ir, jeigu ji nėra užimta, leidžia vartotojui sukurti skelbimą priemonei parduoti.
\subsubsection{Alternatyvus scenarijus(Priemonė, kurią norima parduoti, užimta)}
	Vartotojas pasirenka skiltį "Technikos sekimas realiu laiku", sistema parodo sąrašą, kuriame rodoma ūkio technika, jos užimtumas, būklė, darbo tvarkaraštis. Vartotojas pasirenka parduoti technikos priemonę, sistema patikrina, ar ši priemonė yra laisva, tačiau ji yra užimta. Sistema vartotojui leidžia pakeisti šios priemonės tvarkaraštį ir nustatyti, jog jo daugiau nebūtų galima pildyti. Sistema tuomet leidžia vartotojui sukurti skelbimą priemonei parduoti.
\subsubsection{Alternatyvus scenarijus(Priemonė, kurią norima parduoti, jau parduodama)}
	Vartotojas pasirenka skiltį "Technikos sekimas realiu laiku", sistema parodo sąrašą, kuriame rodoma ūkio technika, jos užimtumas, būklė, darbo tvarkaraštis. Vartotojas pasirenka parduoti technikos priemonę, sistema patikrina, ar ši priemonė yra laisva, tačiau jau yra sukurtas skelbimas jai parduoti. Vartotojui išvedamas klaidos pranešimas ir jis nuvedamas į ūkio technikos sąrašą.
\subsubsection{Funkciniai reikalavimai}
\begin{table}[htbp]
	\begin{tabularx}{1\textwidth}{ |P{2.5cm}|X|P{3cm }| }  \hline
           	Nr. & Reikalavimas &  Prioritetas (1-10)  \\   \hline 
         		FR-4.01 & Sistema leidžia vartotojui parduoti techniką & 10  \\   \hline
		FR-4.01.01 & Patikrinti, ar nėra jau sukurto skelbimo priemonei & 8 \\  \hline
		FR-4.01.02 & Patikrinti, ar ši priemonė nėra jau parduodama & 8 \\ \hline
		FR-4.01.03 & Sukurti skelbimus skelbimus internete & 8 \\ \hline
	\end{tabularx}
\end{table}

\subsection{Žemės parametrų sekimas}
\subsection{Orų prognozės sekimas}
\subsection{Žemės parametrų spėjimas}
\subsection{Gyvūnų maisto išteklių sekimas}
\subsection{Automatinis gyvūnų maitinimas}
\subsection{Automatinis maisto užsakymas}
\subsection{Ūkio technikos sekimas realiu laiku}
\subsection{Ūkio technikos valdymas nuotolinių būdu realiu laiku}
\subsection{Autonominis ūkio technikos veikimas}
\subsection{Ūkininko valdomos teritorijos žymėjimas sutartiniais ženklais}
\subsection{Ligų žemėlapis kaimyninėse teritorijose}

\subsection{Sąskaitų išrašymas}
	\subsubsection{Pagrindinis scenarijus}
	Vartotojas pasirenka skiltį "Sąskaitos". Atsidariusiame lange vartotojui parodo sąrašą visų sąskaitų kurias jam reikia apmokėti. Šalia kiekvienos sąskaitos yra mygtukas, kurį paspaudus vartotojas nukreipiamas į apmokėjimo platformą pagal vrtotojo pasirinkimą kurioje jis gali apmokėti konkrečia sąskaitą. Pagrindiniame sąskaitų lange vartotojas gali pasižiūrėti kokia sąskaitos suma ir kada ji buvo išrašyta. Taip pat gal pasižiūrėti jau apmokėtų sąskaitų istorija.
	\subsubsection{Alternatyvus scenarijus(Nėra sąskaitų kurias reiktų apmokėti)}
	Jeigu sistemoje nėra sąskaitų kurias vartotojas turėtų apmokėti, vartotojui paspaudus ant "Sąskaitos" mygtuko jam bus parodytas informacinis langas pranešantis, kad nėra sąskaitų kurias šiuo metu reiktu apmoketi. Lange bus galima pasirinkti arba eitį į pagrindinį meniu arba peržiūrėti sąskaitų istorija.
	\subsubsection{Alternatyvus scenarijus(Nėra jau apmokėtų sąskaitų)}
	jeigu vartotojas "Sąskaitos" lange bandys peržiūrėti sąskaitų istorija ir joje nieko nebus, vartotojui bus parodomas informacinis langas kuriame bus pranešimas "Sąskaitų istorijos nėra" ir vartotojas bus nukreipiamas į sąskaitas kurias reikia apmokėti arba į alternatyvų scenarijų kai nėra sąskaitų kurias reiktų apmokėti.
	\subsubsection{Alternatyvus scenarijus(Negalima pasiekti pasirinktos apmokėjimo sistemos)}
	Vartotojui bandant apmokėti savo sąskaitas jis gali pasirinkti iš keletos apmokėjimo platformų. Jeigu jo pasirinkta apmokėjimo platforma atmeta vartotojo apmokėjimo prašymą dėl nepakankamų lėšų, neveikiančios apmokėjimo sistemos ar kitų nenumatytų nesklandumų sistema parodys informacini pranešimą "Sąskaitos apmokėti nepavyko". Po pranešimu sistema parodys altertnatyvius apmokėjimo būdus, o jeigų tokių būdų neegzistuos sistema parodys apmokėjimo informaciją su kuria vartotojas sąskaitą galės apmokėti pats, neautomatiškai.
	\subsubsection{Funkciniai reikalavimai}
\begin{table}[htbp]
	\begin{tabularx}{1\textwidth}{ |P{2.5cm}|X|P{3cm }| }  \hline
		Nr. & Reikalavimas & Prioritetas(1-10) \\ \hline
		FR-14.01 & Sistema leidžia peržiūrėti sąskaitas & 10 \\ \hline
		FR-14.01.01& Peržiūrėti sąskaitas kurias reikia apmokėti & 10 \\ \hline
		FR-14.01.02 & Peržiūrėti apmokėtų sąskaitų istoriją & 7 \\ \hline 
		FR-14.01.03 & Laikyti sąskaitų istoriją ilgiau negu pusę metų & 5 \\ \hline
		FR-14.02 & Amokėti sąskaitą & 7 \\ \hline
		FR-14.02.01 & Pateikti apmokėjimo platformų sąrašą & 9 \\ \hline
		FR-14.02.02 & Suteikti apmokėjimo informaciją & 10 \\ \hline
	\end{tabularx}
\end{table}
\subsection{Darbuotojų samdymas}
	\subsubsection{Pagrindinis scenarijus}
	Vartotojas pasirenka skiltį "Darbuotojų samdymas". Vartotojui parodomas langas su dviem pasirinkimais: "Dėti darbo skelbimą" ir "Žiūrėti darbo skelbimus". Pasirinkus pirmajį pasirinkima vartotojas nukreipiamas į formą kurią vartotojas turi užpildyti apie norimą darbuotoją. Paspaudus antrajį pasirinkimą vartotojas nukreipiamas į darbo skelbimus kurie jau yra internete.
	\subsubsection{Alternatyvus scenarijus(Neatsidaro darbo skelbimų puslapis)}
	Vartotojas bando atidaryti darbo skelbimų puslapį spausdamas "Žiūrėti darbo skelbimus". Nepavykus pateikti puslapio vartotojui ekrane atsiranda informacinis pranešimas, kad darbo skelbimų parodyti nepavyko ir vartotojas perkels į pagrindinį programos puslapį.
	\subsubsection{Alternatyvus scenarijus(Neteisingai suvesti duomenys bandant įdėti darbo skelbimą)}
	Vartotojas bando užpildyti formą ir įdėti skelbimą ir padaro įvedimo klaidą(pvz. į siūlomos algos vietą įveda raides) programa pažymi vietą kur vartotojas padarė klaidą ir neleidžia išsiųsti skelbimo kol formoje yra klaidų.
	\subsubsection{Funkciniai reikalavimai}
\begin{table}[htbp]
	\begin{tabularx}{1\textwidth}{ |P{2.5cm}|X|P{3cm }| }  \hline
		Nr. & Reikalavimas & Prioritetas(1-10) \\ \hline
		FR-15.01 & Sistema leidžia peržiūrėti darbo skelbimų sąrašą & 9 \\ \hline
		FR-15.01.01 & Surušiuoti darbuotojus pagal lytį & 7 \\ \hline
		FR-15.01.02 & Surušiuoti darbuotojus pagal amžių & 7 \\ \hline
		FR-15.01.03 & Surušiuoti darbuotojus pagal specialybę & 8  \\ \hline
		FR-15.01.04 & Surušiuoti darbuotojus pagal darbo patirtį & 8 \\ \hline
		FR-15.01.05 & Surušiuoti darbuotojus pagal gyvenamają vietą & 7 \\ \hline
		FR-15.01.06 & Peržiūrėti darbuotojų gyvenimo aprašymus & 9 \\ \hline
		FR-15.01.07 & Peržiūrėti darbuotojų motyvacinius laiškus & 5 \\ \hline
		FR-15.02 & Sistema leidžia užpildyti darbo skelbimą & 9 \\ \hline
		FR-15.02.01 & Pasirinkti vieną ar kelias skelbimų publikavimo platformų & 8 \\ \hline
		FR-15.02.02 & Ištrinti darbo skelbimą & 9 \\ \hline
		FR-15.02.03 & Koreguoti darbo skelbimą & 8 \\ \hline
	\end{tabularx}
\end{table}	
	
	
\subsection{Potencialaus pelno skaičiavimas}
	\subsubsection{Pagrindinis scenarijus}
	Vartotojas pasirenka "Potencialaus pelno skaičiavimas" sistemoje. Programa įjungia langą kuriame yra sąrašas visų šiuo metu vartotojo turimų pardavimui skirtų resursų. Vartotojas gali pasirinkti žiūrėti bendrą pelną kurį gautų pardavęs resursus arba pasirinkti tam tikrus resursus kuriuos norėtų parduoti ir kokį kiekį norėtų parduoti. Pagal vartotojo pasirinkimą ir rinkos kainą yra apskaičiuojamas potencialus pelnas.
	\subsubsection{Alternatyvus scenarijus(Nartotojas neturi parduodamų resursų)}
	Vartotojas pasirenka "Potencialaus pelno skaičiavimas" skilį. Sistemoje nėra užregistruotą jokių resursų kuriuos vartotojas galėtų parduoti. Vartotojui parodomas pranešimas apie nepavykusią opeaciją dėl resursų trūkumo. Programa įjungia pagrindinį langą. 
	\subsection{Alternatyvus scenarijus(Neveikia išoriniai servisai suteikiantys informaciją apie rinkos kainas)}
	Vartotojas pasirenka "Potencialaus pelno skaičiavimas" skilį. Norint apskaičiuoti potencialų pelną naudojamas išorinis servisas nustatyti rinkos kainą. Šiam servisui neveikiant programa vartotojui parodo informacinį pranešimą dėl nesėkmingo bandymo susisiekti su išoriniu servisu. Tokiu atveju programa naudoją naujausią turėtą rinkos kainą pelno skaičiavimui. Dėl senų duomenų naudojimo ir galimų netikslumų vartotojas taip pat informajamas informacine žinute.
	\subsubsection{Funkciniai reikalavimai}
\begin{table}[htbp]
	\begin{tabularx}{1\textwidth}{ |P{2.5cm}|X|P{3cm }| }  \hline
		Nr. & Reikalavimas & Prioritetas(1-10) \\ \hline
		FR-16.01 & Sistema pateikia vartotojui potencialų pelną & 10 \\ \hline
		FR-16.01.01 & Pasirinkti kuriuos resursus skaičiuoti & 8 \\ \hline
		FR-16.01.02 & Pasirinkti resursu kiekį skaičiavimui & 8 \\ \hline
		FR-16.01.03 & Pasirinkti pagal kuriuos rinkos duomenis skaičiuoti pelną & 8 \\ \hline
	\end{tabularx}
\end{table}	
	 
\subsection{Derliaus sekimas}
	\subsubsection{Pagrindinis scenarijus}
	Vartotojas sistemoje pasirenka "Derliaus sekimas". Programa įjungią vartotojo turimų resursų sąrašą. Saraše prie kiekvieno resurso parašyta kada jis buvo gautas ir kokia kiekvieno resurso galiojimo trukmė. Paspaudus ant konkretaus resurso pasirodo langas, kuriame pateikta detalesnė informacija konkretų resursą. Detali informacija susideda iš resurso gavimo laiko, darbuotojų kurie dirbo prie konkretaus resurso gavimo, kiek pinigų vartotojas gautų pardavęs resursus rinkos kaina.
	\subsubsection{Alternatyvus scenarijus(Vartotojas neturi resursų kuriuos būtų galima parodyti)}
	Vartotojas sistemoje pasirenka "Derliaus  sekimas". Sistemoje nėra užregistruota jokių resursų. Vartotojui parodomas informacinis pranešimas apie tai kad sistemoje nėra registruotų resursų. Vartotojas sukreipiamas į pagrindinį langą.
	\subsubsection{Alternatyvus scenarijus(Neveikia rinkos skaičiavimo funkcija)}
	Vartotojas sistemoje pasirenka "Derliaus sekimas". Sistema atidaro langą su resursais. Vartotojas pasirenka konkretų resursą norėdamas sužinoti detalesnią informaciją. Rinkos kainos funkciją dėl tam tikrų priežasčių neveikia ir neįmanoma parodyti tikslios rinkos kainos. Vartotojui parodomas informacinis pranešimas dėl netikslios kainos. Taip pat vartotojas informuojamas, kad preliminari kaina bus skaičiuojama naudojantis senais rinkos duomenimis.
	\subsubsection{Funkciniai reikalavimai}
\begin{table}[htbp]
	\begin{tabularx}{1\textwidth}{ |P{2.5cm}|X|P{3cm }| }  \hline
		Nr. & Reikalavimas & Prioritetas(1-10) \\ \hline
		FR-17.01 & Sistema vartotojui pateikia turimų resursų sąraša & 10 \\ \hline
		FR-17.01.01 & Pateikti turimų resursų kiekį & 8 \\ \hline
		FR-17.01.02 & Pateiktį turimų resursų galiojimo laiką & 8 \\ \hline
		FR-17.01.03 & Pateikti laiką kada buvo gautas konkretus resursas & 8 \\ \hline
		FR-17.01.04 & Pateikti sąrašą žmonių kurie dirbo prie konkretaus resurso gavimo & 6 \\ \hline
		FR-17.01.05 & Pateikti resurso rinkos kainą & 7 \\ \hline 
		FR-17.01.06 & Pateikti kiek pinigų vartotojas gautų pardavęs konkretų resursą, pagal rinkos kainą & 7 \\ \hline
	\end{tabularx}
\end{table}	
	
\subsection{Buhalterijos tvarkymas}
	\subsubsection{Pagrindinis scenarijus}
	Vartotojas pasirenka "Buhalterijos tvarkymas" pagrindiniame meniu. Programa atidaro langą kuriame vartotojas nukreipiamas į buhalterijos tvarkymo platforma. Vartotojas prisijungia prie platformos naudodamas savo duomenis.
	\subsubsection{Alternatyvus scenarijus(Neveikia buhalterijos servisas)}
	Vartotojas pasirenka "Buhalterijos tvarkymas". Išorinis buhalterijos servisas neveikia. Vartotojui parodomas in formacinis pranešimas, kad dėl tam tikrų priežasčių buhalterijos tvarkymo servisas šiuo metu neveikia. Vartotojas kureipiamas į darbalaukio aplikaciją kurioje ji gali tvarkyti buhalteriją
	\subsubsection{Funkciniai reikalavimai}
	\begin{table}[htbp]
	\begin{tabularx}{1\textwidth}{ |P{2.5cm}|X|P{3cm }| }  \hline
		Nr. & Reikalavimas & Prioritetas(1-10) \\ \hline
		FR-18.01 & Sistema suteikia vartotojui prieeiga prie buhalterijos tvarkymo platformos & 9 \\ \hline
	\end{tabularx}
\end{table}
\subsection{Rinkos kainų sekimas}
	\subsubsection{Pagrindinis scenarijus}
	Vartotojas pagrindiniame lange pasirenka "Rinkos kainų sekimas". Programa atidaro langą langą kuriame rodos resursų kainos realiu laiku. Resursų kainos paterikiamos diagramos pavidalu. Vartotojas gali pasirinkti laiko intervalą kuriame nori stebėti rinkos kainos pokyčius ir kurių resursų kainas stebėti.
	\subsubsection{Atlernatyvus scenarijus(Nepavyksta gauti rinkos kainų)}
	Vartotojas pagrindiniame lange pasirenka "Rinkos kainų sekimas". Dėl tam tikrų priežasčių nepavyksta gauti dabartinės rinkos kainos. Vartotojui parodomas informacinis pranešimas dėl netikslių rinkos duomenų. Vartotojui parodoma rinkos kainų istoriją kuri yra saugoma sistemoje.
	\subsubsection{Alternatyvus scenarijus(Nepavyksta gauti rinkos kainų istorijos)}
	Vartotojas pagrindiniame lange pasirenka "Rinkos kainų sekimas". Dėl nenumatytų nesklandumų nepavyksta gauti dabartinių rinkos kainų ir sistemoje nėra išsaugota kainų istorija. Vartotojas informuojamas, kad nepavyksta gauti rinkos kainų realiu laiku ir kad sistemoje nėra išsaugotos kainų istorijos. Vartotojas nukreipiamas  į pagrindinį programos langą.
	\subsubsection{Funkciniai reikalavimai}
\begin{table}[htbp]
	\begin{tabularx}{1\textwidth}{ |P{2.5cm}|X|P{3cm }| } \hline
		Nr. & Reikalavimas & Prioritetas(1-10) \\ \hline
		FR-19.01 & Sistema parodo rinkos kainas realius laiku &  9 \\ \hline
		FR-19.01.01 & Rodyti diagramą & 8 \\ \hline
		FR-19.01.02 & Leisti pasirinkti konkrečius resursus jų kainos rodymui & 8 \\ \hline
		FR-19.01.03 & Rodyti rinkos kainų istoriją & 7 \\ \hline
		FR-19.01.04 & Saugoti rinkos kainų istoriją & 7 \\ \hline
		
	\end{tabularx}
\end{table}
	 
\subsection{Automatinis žemės laistymas}
\subsection{Pagalbos iškvietimas}
	\subsubsection{Pagrindinis scenarijus}
	Vartotojas pasirenka ''Pagalbos iškvietimas'' pagrindiniame meniu. Programa atidaro langą, kuriama vartotojas pasirenka reikiamos pagalbos tipą, sistema tada nusiunčia numatytąjį pranešimą su prašymų atvykti, parodo gautą atsakymą.
	\subsubsection{Alternatyvus scenarijus(nėra interneto ryšio)}
	Vartotojas pasirenka ''Pagalbos iškvietimas'' pagrindiniame meniu. Programa atidaro langą, kuriama vartotojas pasirenka reikiamos pagalbos tipą, sistema nepavykus išsiųsti pranešimo ji parodo pranešimą, kuriame nurodo, kokiu numeriu reikia paskambinti norint išsikviesti norimą pagalbą.
	\subsubsection{Funkciniai reikalavimai}
	\begin{table}[htbp]
		\begin{tabularx}{1\textwidth}{ |P{2.5cm}|X|P{3cm }| }  \hline
			Nr. & Reikalavimas & Prioritetas(1-10) \\ \hline
			FR-21.01 & Sistema leidžia pasirinkti reikiamos pagalbos tipą & 10 \\ \hline
			FR-21.01.01 & Veterinarą & 9 \\ \hline
			FR-21.01.02 & Agronomą & 8 \\ \hline
			FR-21.01.03 & Greitają pagalbą & 10 \\ \hline
			FR-21.01.04 & Gaisrinę & 10 \\ \hline
			FR-21-02 & Vartotojui pasirinkus sistema iškviečia pasirinktą pagalbą & 10 \\ \hline	
			FR-21-02.01 & Parodo gautą atsakymą iš pagalbos suteikėjo & 9 \\ \hline
			FR-21-03 & Sistema leidžia nustatyti numatytąjį pranešimą siunčiamą kiekvienai pagalbai & 9 \\ \hline	
			FR-21-03 & Nesant galimybės nusiųsti pranešimo sistema parodo atitinkamos pagalbos kontaktus & 10 \\ \hline	
			FR-21-03.01 & Tel. numeri & 10 \\ \hline	
			FR-21-03.02 & Adresą & 8 \\ \hline	
			FR-21-03.03 & El. paštą & 6 \\ \hline	
						
		\end{tabularx}
	\end{table}
\subsection{Ataskaitos apie ūkį sudarymas}
	\subsubsection{Pagrindinis scenarijus}
	Vartotojas pasirenka ''Ūkio ataskaita'' pagrindiniame meniu. Programa atidaro langą, kuriame vartotojas pasirenka norimą laikotarpį, ir peržiūri to laikotarpio ataskaitą.
	\subsubsection{Alternatyvus scenarijus(Nepasiekiama duomenų bazė)}
	Vartotojas pasirenka ''Ūkio ataskaita'' pagrindiniame meniu. Programa atidaro langą, kuriame vartotojas pasirenka noriąo laikotarpį, programa išveda klaidos pranešimą ir pasiūlo pamėginti vėliau.
	\subsubsection{Funkciniai reikalavimai}
	\begin{table}[htbp]
		\begin{tabularx}{1\textwidth}{ |P{2.5cm}|X|P{3cm }| }  \hline
			Nr. & Reikalavimas & Prioritetas(1-10) \\ \hline
			FR-22.01 & Sistema leidžia pasirinkti norimą laikotarpį ataskaitai & 9 \\ \hline
			FR-22-02 & Sistema sudaro ataskaitą vartotojo pasirinktam laikotarpiui & 10 \\ \hline	
		\end{tabularx}
	\end{table}
\subsection{Darbų sąsašo sudarymas}
\subsection{Žolių ir ligų katalogas}
	\subsubsection{Pagrindinis scenarijus}
	Vartotojas pasirenka ''Žolių ir ligų katalogas'' pagrindiniame meniu. Programa atidaro langą, kuriame vartotojas gali matyti vaistinių žolelių arba ligų katalogą.
	\subsubsection{Alternatyvus scenarijus(Nesėkminga paieška)}
	Vartotojas pasirenka ''Ūkio ataskaita'' pagrindiniame meniu. Programa atidaro langą, kuriame parodomas katalogas, vartotojui atlikus paiešką ir neradus jokių rezultatų išvedamas pranešimas, kad tokių duomenų kataloge nėra.
	\subsubsection{Funkciniai reikalavimai}
	\begin{table}[htbp]
		\begin{tabularx}{1\textwidth}{ |P{2.5cm}|X|P{3cm }| }  \hline
			Nr. & Reikalavimas & Prioritetas(1-10) \\ \hline
			FR-24.01 & Sistema leidžia pasirinkti norimą katalogą & 10 \\ \hline
			FR-24.01.01 & Žolelių & 10 \\ \hline
			FR-24.01.02 & Gyvūnų ligų & 9 \\ \hline
			FR-24.01.03 & Žmonių ligų & 7 \\ \hline
			FR-24.02 & Sistema leidžia atlikti paiešką  & 9 \\ \hline
			FR-24.02.01 & Pagal simbolių eilute esančia pavadinime & 9 \\ \hline
			FR-24.02.01 & Pagal simbolių eilute esančia aprašyme & 8 \\ \hline
			FR-24.03 & Sistema leidžia norimą katalogą rikiuoti & 9 \\ \hline
			FR-24.03.01 & Pagal alfabetinę tvarką  & 9 \\ \hline
		\end{tabularx}
	\end{table}
\subsection{Ūkio chemijos sąrašas su aprašymais ir kainomis}

\section{Kiti nefunkciniai reikalavimai}
\subsection{Našumo reikalavimai}
\subsection{Saugos reikalavimai}
\subsection{Saugumo reikalavimai}
\subsection{Sistemos kokybės atributai}
\subsection{Biznio taisyklės}

\section{Kiti reikalavimai}
\sectionnonum{Žodynas}
\sectionnonum{Analizės modeliai}
\end{document}