\documentclass[oneside]{VUMIFPSkursinis}
\usepackage{algorithmicx}
\usepackage{algorithm}
\usepackage{algpseudocode}
\usepackage{amsfonts}
\usepackage{float}
\usepackage{amsmath}
\usepackage{bm}
\usepackage{caption}
\usepackage{color}
\usepackage{float}
\usepackage{graphicx}
\usepackage{listings}
\usepackage{subfig}
\usepackage{tabularx}
\usepackage{wrapfig}
\usepackage[%  
    colorlinks=true,
    linkcolor=black
]{hyperref}
\university{Vilniaus universitetas}
\faculty{Matematikos ir informatikos fakultetas}
\department{Programų sistemų katedra}
\papertype{Laboratorinis darbas II}
\title{Automatinė ūkio valdymo sistema}
\titleineng{Automatic farm management system}
\status{2 kurso 3 grupės studentai}
\author{Matas Savickis}
\secondauthor{Justas Tvarijonas}  
\thirdauthor{Greta Pyrantaitė}   
\fourthauthor{Rytautas Kvašinskas}
\supervisor{Karolis Petrauskas, Doc., Dr.}
\date{Vilnius – \the\year}


\bibliography{bibliografija}

\begin{document}
\maketitle

\tableofcontents
\centering

\section{Įvadas}
\subsection{Tikslas}
\subsection{Dokumento skaitytojai}
\begin{itemize}
	\item Užsakovas
	\item Projekto vadovas
	\item Projektuotojas
	\item Testuotojas
	\item Teisininkas
\end{itemize}

\section{Išoriniai sąsajos reikalavimai}
\subsection{Vartotojo sąsajos reikalavimai}

\section{Funkciniai reikalavimai}
\subsection{Duomenų suvedimas}
\subsection{Duomenų peržiūra}
\subsection{Vartotojo registracija}
\newcolumntype{R}{>{\raggedleft\arraybackslash}X}%
\begin{tabularx}{\textwidth}{ |X|X|X|X| }
  \hline
  Numeris & Reikalavimas & Prioritetas \newline (1-10)  & Egzistavimo \newline laikas \\
  \hline 
 **  & Darbuotojui turi būtų leista pateikti užklausą registruotis  & 10  & ?  \\
  \hline
   **  & Darbuotojui turi būtų leista Užpildyti registracijos formą  & 10  & ?  \\
  \hline
     **  & Darbuotojui turi būtų leista patvirtinti registraciją  & 10  & ?  \\
  \hline
\end{tabularx}




\section{Nefunkciniai reikalavimai}
\subsection{Naudojimo saugos reikalavimai}
\subsection{Duomenų saugos reikalavimai}
\subsection{Kokybės reikalavimai}
\subsection{Veikimo principai}





\end{document}
