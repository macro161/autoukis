\documentclass[oneside]{VUMIFPSkursinis}
\usepackage{algorithmicx}
\usepackage{algorithm}
\usepackage{algpseudocode}
\usepackage{amsfonts}
\usepackage{float}
\usepackage{amsmath}
\usepackage{bm}
\usepackage{caption}
\usepackage{color}
\usepackage{float}
\usepackage{graphicx}
\usepackage{listings}
\usepackage{subfig}
\usepackage{tabularx}
\usepackage{wrapfig}
\newcolumntype{P}[1]{>{\centering\arraybackslash}p{#1}}
\usepackage[%  
    colorlinks=true,
    linkcolor=black
]{hyperref}
\university{Vilniaus universitetas}
\faculty{Matematikos ir informatikos fakultetas}
\department{Programų sistemų katedra}
\papertype{Laboratorinis darbas III}
\title{Automatinė ūkio valdymo sistema}
\titleineng{Automatic Farm Management System}
\status{2 kurso 3 grupės studentai}
\author{Matas Savickis}
\secondauthor{Justas Tvarijonas}  
\thirdauthor{Greta Pyrantaitė}   
\fourthauthor{Rytautas Kvašinskas}
\supervisor{Karolis Petrauskas, Doc., Dr.}
\date{Vilnius – \the\year}


\bibliography{bibliografija}

\begin{document}
\maketitle
\centering
\tableofcontents


\section{Įvadas}
Pasaulyje technologijos vystosi labai sparčiai. Prognozuojama, kad ateityje robotizacija pakeis žmonių darbo jėgą ir žmonių darbo jėga bus nebereikalinga. Kolkas pasaulio ir Lietuvos ūkio sektoriuje robotizacija vykdoma minimaliai. Mūsų kuriama sistema siekia įnešti šią ūkio automatizaciją į Lietuvos rinką. Sistemos tikslas automatizuoti kuo daugiau ūkio veiklų. Šio laboratorinio darbo tiksalas atlikti verslo analize: atlikti išorinę ir vidinę procesų analizę, atlikti SWOT analizę, sukurti verslo vystymo strategiją ir išsiaiškinti ar turint dabartinę rinkos, technologijos ir žinių situacija ar sistemos kūrimas ir implementavimas pasiteisintu kaip verslo planas.

\section{Išorinė analizė}
	\subsection{Verslo įeiga/išeiga}
	\subsection{Efektyvumas}
	\subsection{Grėsmės/galimybės}
\section{Vidinė dalykinės srities analizė}
	\subsection{Veiklos principai}
	\subsection{Rodikliai}
\section{Analizės rezultatai, apibendrinimas}
	\subsection{Stiprybės}
	\subsection{Silpnybės}
	\subsection{Galimybės}
	\subsection{Pavojai}
\section{Vizija, misija}
\section{Strateginiai tikslai}
\section{Sistemos naudojimo scenarijus}
\section{Sistemos įgyvendinimo planas}
\section{Įgyvendinamumo analizė}
	\subsection{Operacinė analizė}
	\subsection{Techninė analizė}
	\subsection{Ekonominio įgyvendinamumo analizė}
	\subsection{Teisinė analizė}
\section{Išvados}
\section{Žodynas}

\end{document}