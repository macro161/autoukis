\documentclass[oneside]{VUMIFPSkursinis}
\usepackage{algorithmicx}
\usepackage{algorithm}
\usepackage{algpseudocode}
\usepackage{amsfonts}
\usepackage{float}
\usepackage{amsmath}
\usepackage{bm}
\usepackage{caption}
\usepackage{color}
\usepackage{float}
\usepackage{graphicx}
\usepackage{listings}
\usepackage{subfig}
\usepackage{tabularx}
\usepackage{wrapfig}
\newcolumntype{P}[1]{>{\centering\arraybackslash}p{#1}}
\usepackage[%  
    colorlinks=true,
    linkcolor=black
]{hyperref}
\university{Vilniaus universitetas}
\faculty{Matematikos ir informatikos fakultetas}
\department{Programų sistemų katedra}
\papertype{Laboratorinis darbas III}
\title{Automatinė ūkio valdymo sistema}
\titleineng{Automatic Farm Management System}
\status{2 kurso 3 grupės studentai}
\author{Matas Savickis}
\secondauthor{Justas Tvarijonas}  
\thirdauthor{Greta Pyrantaitė}   
\fourthauthor{Rytautas Kvašinskas}
\supervisor{Karolis Petrauskas, Doc., Dr.}
\date{Vilnius – \the\year}


\bibliography{bibliografija}

\begin{document}
\maketitle
\centering
\tableofcontents


\section{Įvadas}
Šia dokumente pateikiama Automatinės ūkio valdymo sistemos(toliau Auto ūkis) verslo analizę. Autoūkis yra sistema suteikianti vartotojui galimybę automatizuoti žemės ūkio valdymą. Sistema suteikia galimybę žemės ūkio techniką: pirkti, parduoti, automatikšai valdyti. Mūsų sistema galima kontroliuoti žemės plotus, juos laistyti bei sekti dirvos parametrus. Vartotojui taip pat suteikiama galimybė samdyti bei atelisti darbuotojus, išrašinėti sąskaitas bei užsiimti kitaip buhalteriniais veiksmais. Tarp panaudojimo galimybių yra ir gyvūnų sekimas realiu laiku, jų informacijos surašymas, bei automatinis gyvūnų maitinimas. Ūkininkas taip pat gali registruoti savo derlių, stebėti rinko ir planuoti būsimą pelną. Visa sistema yra moduli kas reiškia, kad sistema yra išskirstytą į keletą atskirų fragmentų ir jų visų potencialiam pirkėjui įsigyti nereikia. Sistemos kūrėjai yra atsakingi už sistemos įdiegimą, bet ne palaikymą. Sistemą turi palaikyti pats pirkėjas arba samdyti mūsų pačių apmokytus technikus.

\section{Išorinė analizė}
	\subsection{Verslo įeiga/išeiga}
	\subsection{Efektyvumas}
	\subsection{Grėsmės/galimybės}
\section{Vidinė dalykinės srities analizė}
	\subsection{Veiklos principai}
	\subsection{Rodikliai}
\section{Analizės rezultatai, apibendrinimas}
	\subsection{Stiprybės}
	\subsection{Silpnybės}
	\subsection{Galimybės}
	\subsection{Pavojai}
\section{Vizija, misija}
\section{Strateginiai tikslai}
\section{Sistemos naudojimo scenarijus}
\section{Sistemos įgyvendinimo planas}
\section{Įgyvendinamumo analizė}
	\subsection{Operacinė analizė}
	\subsection{Techninė analizė}
	\subsection{Ekonominio įgyvendinamumo analizė}
	\subsection{Teisinė analizė}
\section{Išvados}
\section{Žodynas}

\end{document}