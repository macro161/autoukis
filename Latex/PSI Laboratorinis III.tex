\documentclass[oneside]{VUMIFPSkursinis}
\usepackage{algorithmicx}
\usepackage{algorithm}
\usepackage{algpseudocode}
\usepackage{amsfonts}
\usepackage{float}
\usepackage{amsmath}
\usepackage{bm}
\usepackage{caption}
\usepackage{color}
\usepackage{float}
\usepackage{graphicx}
\usepackage{listings}
\usepackage{subfig}
\usepackage{tabularx}
\usepackage{wrapfig}
\newcolumntype{P}[1]{>{\centering\arraybackslash}p{#1}}
\usepackage[%  
    colorlinks=true,
    linkcolor=black
]{hyperref}
\university{Vilniaus universitetas}
\faculty{Matematikos ir informatikos fakultetas}
\department{Programų sistemų katedra}
\papertype{Laboratorinis darbas III}
\title{Automatinė ūkio valdymo sistema}
\titleineng{Automatic Farm Management System}
\status{2 kurso 3 grupės studentai}
\author{Matas Savickis}
\secondauthor{Justas Tvarijonas}  
\thirdauthor{Greta Pyrantaitė}   
\fourthauthor{Rytautas Kvašinskas}
\supervisor{Karolis Petrauskas, Doc., Dr.}
\date{Vilnius – \the\year}


\bibliography{bibliografija}

\begin{document}
\maketitle
\centering
\tableofcontents


\section{Įvadas}
Pasaulyje technologijos vystosi labai sparčiai. Prognozuojama, kad ateityje robotizacija pakeis žmonių darbo jėgą ir žmonių darbo jėga bus nebereikalinga. Kolkas pasaulio ir Lietuvos ūkio sektoriuje robotizacija vykdoma minimaliai. Mūsų kuriama sistema siekia įnešti šią ūkio automatizaciją į Lietuvos rinką. Sistemos tikslas automatizuoti kuo daugiau ūkio veiklų. Šio laboratorinio darbo tiksalas atlikti verslo analize: atlikti išorinę ir vidinę procesų analizę, atlikti SWOT analizę, sukurti verslo vystymo strategiją ir išsiaiškinti ar turint dabartinę rinkos, technologijos ir žinių situacija ar sistemos kūrimas ir implementavimas pasiteisintu kaip verslo planas.
\section{Verslo proceso aprašas}

\section{Išorinė analizė}
	\subsection{Black Box analizė}
\begin{figure}[H]
		\centering	
	\includegraphics[width=18cm,height=20cm,keepaspectratio]{BlackBox.png}
	\caption{Tiekimo grandinė}
	\label{fig:supplyChain}
\end{figure}
	\begin{itemize}
		\item{Įeiga: }
		\begin{itemize}
			\item{1. Žemės drekinimo valdymas - vartotojas pasirenka žemės drekinimo funkcija rankiniu arba automatiniu būdu taip užtikrindamas optimalų žemės parametrų palaikymą}
			\item{2. Gyvūnų maitinimo valdymas - vartotojas pasirenka gyvūnų maitinimą automatiniu arba rankiniu būdu taip užtikrindamas gretą, efektyvų ir humanišką gyvūnų rūpinimasi net ir tuo atveju kai nepavyksta pasiekti gyvūnų laiku dėl iškimulių kliučių}
			\item{3. Žemės parametrų sekimas - vartotojui suteikiama galimybė setėti įvairius žemės parametrus ir užtikrinti reikiamą dirvos priežiūrą ir sveikatą}
			\item{4. Buhalterijos tvarkymas - Vartotojui suteikiama galimybė tvarkyti savo buhaleterinis reikalus paprastu ir suprantamu interfeisu}
			\item{5. Resursų sekimas - vartotojas turi galimybę sekti savo turimus resursus, jų rinkos kainą ir užtikrinti maksimalų pardavimo pelną}
			\item{6. Gyvūnų sekimas - sistema užtikrina galimybę sekti kiekvieną individualų gyvūną ir pabegimo atvejų jį nesunkiai surasti}
			\item{7. Ūkio technikos valdymas - sistema leidzia vartotojui valdyti ūkio technika nuotoliniu būdu taip išvengiant žmogikos klaidos ir padarant visa darbą efektyvesniu}
			\item{8. Teritorijos valdymas - sistmeoje vartotojas gali žymėti savo teritorija sutariniais ženklais taip palentgvindamas navigaciją po teritoriją}
			\item{9. Orų sekimas - sistema pateikia vartotojui orų prognozes kurios užtikrintų vartotoja kaip teisingai reiktų valdyti savo ūki ir atsižvelgti į gamtos veiksnius}
			\item{10. Pagalbos iškvietimas - sistema suteikia galimybę išsikviesti pagalbos tarnybas ištikus nelaimingam atsitikimui.}
		\end{itemize}
		\item{Išeiga:}
		\begin{itemize}
			\item{Sistema užtikriną automatizuotą ūkio procesų valdymą, sumažiną žmogaus įsikišimą į ūkio procesų valdymą taip sumažindama žmogiško faktoriaus klaidas. Potencialiai sistema sumažiną darbuotojų poreikį kas sumažiną reikalingus finansus samdyti darbuotojus bei taisyti jų padarytas klaidas}
		\end{itemize}
		\item{Teisinė bazė: }
		\begin{itemize}
			\item{Kuria ir įgyvendinant sistemą privalkoma atsižvelgti į Lietuvos Respublikos konstituciją, įstatymus bei darbo kodeksą. Jautrausia vieta teisiškai yra gyvūnų priežiūra. Šioje vietoje svarbu atsižvelgti į Lietuvos Respublikos veterinarijos įtatymus kurians automatizuotą gyvūnų maitinimą, kad nebūtų pažeistos gyvūnų teisės }
		\end{itemize}
		\item{Reputacija: }
		\begin{itemize}
			\item{Sienkiant paversti Automatinę ūkio valdymo sistemą populiariu variantu Lietuvos ūkininkms būtina užtikrinti gera mūsų įmonės bei sistemos reputaciją. Gerai veikianti sistema turėtų padaryti ūkininkus laimingais kas skleistų gerą reputaciją apie mūsų sistemą ir padėtų didinti vartotojų bazę. }
		\end{itemize}
	\end{itemize}
	
	

	\subsection{Tiekimo grandinė}
	\begin{itemize}
\item Diagramoje (1 pav.) pavaizduoti pagrindiniai tiekėjai bei pirkėjai, su kuriais bendrauja mūsų nagrinėjamos srities atstovai.
		\begin{figure}[H]
		\centering	
	\includegraphics[width=18cm,height=20cm,keepaspectratio]{supplyChain.png}
	\caption{Tiekimo grandinė}
	\label{fig:supplyChain}
\end{figure}
\end{itemize}
\subsubsection{Pirkėjų apibrėžimas}
Kaip pagrindinius ūkio sukuriamos produkcijos pirkėjus galime įvardinti eilinius žmones, kadangi praktiškai visa ūkio produkcija suvartojama maisto pavidalu. Žinoma dauguma žmonių šią produkciją perka ne tiesiai iš ūkininkų, o iš perpardavinėtojų ar perdirbėjų, kurie ūkio teikiamą produkciją paverčia į jau paruoštą vartoti produktą. Taigi nors iš pirmo žvilgsnio atrodo, kad ūkio produkcijos pirkėjai yra perpardavinėtojai ir perdirbėjai, tačiau tikrieji pirkėjai yra tie, kurie perka galutinį produktą, kadangi į visas darbas yra orientuotas į juos.
\subsubsection{Pirkėjų poreikiai}
Šioje diagramoje matome, kad ūkio atstovai savo produkciją paskirsto 4 keliais, tačiau juos galime padalinti į dvi dalis, ūkio tiekiama produkcija turgaus prekeiviams sudaro labai mažą dalį visos produkcijos, kadangi šią sritį dažniausiai pasirenka tie ūkininkai, kurie turi mažesnį kiekį produkcijos ir jiems mažesnių kiekių pardavinėjimas nesukelia jokių problemų, didesni ūkiai priklausomai nuo jų sukuriamo produkto paprastai renkasi parduotuves bei mėsos arba grūdų supirkėjus, kadangi jie pajėgus produkciją supirkti labai dideliais kiekis taip palengvindami ūkio produkcijos administravimą. Iš šių subjektų, kurie bendrauja su ūkio kokybei didžiausius reikalavimus kelia turgaus prekeiviai bei parduotuvės atstovai, kadangi jie tiesiogiai bendrauja su vartotojais, o vartotojai visada tikisi aukščiausios kokybės produkcijos. O mėsos ir grūdų supirkėjai paprastai užsiima masiniu produkcijos perdirbimu, ko pasekoje šiek tiek prastesnės kokybės produktai jiems nesudaro didelių kliučių pateikti vartotojui tinkamos kokybės produktą.
\subsubsection{Derybinės galimybės su tiekėjais}
Panašiai kaip ir pirkėjus, taip ir tiekėjus galime suskirstyti į dvi pagrindines grupes. Tokie tiekėjai, kaip trašų gamintojai, sėklų pardavinėtojai, meteorologai ar peryklos bei gyvulių pardavėjai neturi didelės derybinės galios, kadangi panašią ar netgi tokią pačią pasiūlą teikiančių tiekėjų yra pakankamai didelis kiekis, todėl patys ūkio atstovai gali išsirinkti jiems geriausius pasiūlymus ir daugelio pasirinkimų. Tačiau visai kita situacija su kitais tiekėjais, tokie tiekėjai kaip sekimo įrangos gamintojai dar galbūt ir neturi labai didelės derybinės galios, tačiau like tiekėjai gali teikti savo reikalavimus, kadangi ūkio atstovai nelabai turi kitų alternatyvų pagrūdintiniems šių tiekimo šakų atstovams, tokia tiekimo šaka kaip autonominių sistemų gamintojai turi maždaug 2-3 didesnius ir klientus patrauklesnius atstovus ir kiekvienas iš jų orientuojasi į šiek tiek kitą pusę, ko pasekoje ūkio atstovai turi rinktis iš daugiausiai 2 tiekėjų, kurių kiekvienas neturi poreikio visomis išgalėmis siekti sutarties nuleidžiant kainas, kadangi jie tiesiog neturi rimtos konkurencijos.
	\subsection{Efektyvumas}
	\subsection{Grėsmės/galimybės}
\section{Vidinė dalykinės srities analizė}
	\subsection{Veiklos principai}
	\subsection{Rodikliai}
\section{Analizės rezultatai, apibendrinimas}
	\subsection{Stiprybės}
	\subsection{Silpnybės}
	\subsection{Galimybės}
	\subsection{Pavojai}
\section{Vizija, misija}
\section{Strateginiai tikslai}
\section{Sistemos naudojimo scenarijus}
\section{Sistemos įgyvendinimo planas}
\section{Įgyvendinamumo analizė}
	\subsection{Operacinė analizė}
	\subsection{Techninė analizė}
	\subsection{Ekonominio įgyvendinamumo analizė}
	\subsection{Teisinė analizė}
\section{Išvados}
\section{Žodynas}

\end{document}