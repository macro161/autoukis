\documentclass{VUMIFPSkursinis}
\usepackage{algorithmicx}
\usepackage{algorithm}



\usepackage{algpseudocode}
\usepackage{amsfonts}
\usepackage{amsmath}
\usepackage{bm}
\usepackage{caption}
\usepackage{color}
\usepackage{float}
\usepackage{graphicx}
\usepackage{listings}
\usepackage{subfig}
\usepackage{wrapfig}









\university{Vilniaus universitetas}
\faculty{Matematikos ir informatikos fakultetas}
\department{Programų sistemų katedra}
\papertype{Laboratorinis darbas I}
\title{Automatinė ūkio valdymo sistema}
\titleineng{Automatic farm management system}
\status{2 kurso 3 grupės studentai}
\author{Matas Savickis}
\secondauthor{Justas Tvarijonas}   % Pridėti antrą autorių
\thirdauthor{Greta Pyrantaitė}   % Pridėti trečią autorių
\fourthauthor{Rytautas Kvasinskas}   % Pridėti ketvirtą autorių
\supervisor{Karolis Petrauskas, Doc., Dr.}
\date{Vilnius – \the\year}

% Nustatymai
% \setmainfont{Palemonas}   % Pakeisti teksto šriftą į Palemonas (turi būti įdiegtas sistemoje)
\bibliography{bibliografija}

\begin{document}
\maketitle

\tableofcontents



\sectionnonum{Įvadas}
Pasirinkome programuoti Automatinę ūkio valdymo sistemą(toliau - Auto ūkis) JAVA programavimo kalba, nes visi komandos nariai turėjo pakankamas žinias programuoti šia kalba ir kalbos funkcionalumas, mūsų manymu, atitiko poreikius. Programuodami pirmą sistemos variantą stengėmės prisilaikyti bendrų objektinio programavimo principų ir JAVA kodo standarto. Apie patį sistemos projektavimą pagalvojome minimaliai dėl tuo metu neturimų žinių. UML diagramų braižymui naudojome www.planttext.com(toliau - PlantText) ir www.draw.io.

\sectionnonum{Žodymas}
\begin{itemize}
	\item Klasės:
		\begin{itemize}
			\item AutoŪkis - pagrindinė(main) programos klasė. Ši klasė piešia grafinę vartotojo sąsaja ir laiko savyje kitų klasių objektus kurių informacija reikalinga piešimui
			\item Map - teritorijos piešimui skirta klasė.
			\item ŽemėsTeritorija - apskaičiuoja tam tikros teritorijos plotą.
 			\item Gyvūnas - klasė skirta gyvūno rodmenims ir metodams saugoti
			\item AriamasLaukas - laiko savyje reikšmes apibūdinančias unikalų lauką ir metodus susijusius su lauko darbu.
			\item Ganykla - laiko parametrus ir metodus darbui su ganyklomis kurios yra žemės plote.
			\item ŪkinisPastatas - saugo ūkinius 
			\item ŪkioTechnika - laiko ūkio technikos charakteristikos reikšmes. Apskaičiuoja technikos judėjimo greitį.
			\item Žemės parametrai - saugo įvairius žemės parametrus(drėgmė, ph...).
			\item Orai - klasė skirta pasiimti orų prognozes iš www.gismeteo.lt kurių paprašo vartotojas.
			\item Žemės detektorius - klasė skirta bendrauti su žemės detektoriumi


		\end{itemize}
	\item Bendri terminai:
		\begin{itemize}
			\item Žemės plotas - vieta kurią valdo ir gali stebėti vartotojas(ūkininkas) 

		\end{itemize}

\end{itemize}




\section{Sukurtos sistemos aprašymas(v1.0)}

\subsection{Loginis pjūvis}
	
 

	
	





\subsection{Kūrimo pjūvis}
\subsection{Proceso pjūvis}
\subsection{Fizinis pjūvis}

\section{Perprojektuotos sistemos aprašymas(To-Be, v2.0)}
\subsection{Loginis pjūvis}
\subsection{Kūrimo pjūvis}
\subsection{Proceso pjūvis}
\subsection{Fizinis pjūvis}


\sectionnonum{Rezultatai ir išvados}












\end{document}
